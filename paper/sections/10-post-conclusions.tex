\documentclass[../main.tex]{subfiles}
\graphicspath{{\subfix{../imagenes/}}}

\begin{document}
\chapter*{Regulatory Framework} \label{chap:regulatory_framework}
\addcontentsline{toc}{chapter}{Regulatory Framework}

The material written in this thesis is published under a Creative Commons license CC BY-NC. The work here can be shared and distributed for non-commercial purposes as long as attribution is properly credited. Adaptations of the work are also permitted as long as they are shared under the same license. The sole copyright and intellectual property belongs to the author. 

Similarly, the software developed for this work is openly available
\footnote{All code developed in this work can be found under the following url: \href{}{https://github.com/simonsanvil/...}, for any questions, concerns, or comments, please contact the author at \href{mailto:simonsviloria@gmail.com}{simonsviloria@gmail.com}
} \change{add url to github repo}
and licensed under an \href{https://en.wikipedia.org/wiki/Apache_License}{Apache License 2.0}, which grants permission to the use, distribution, and modification of the code for commercial and non-commercial purposes (restrictions apply, see license for details). 

Any desire to use this work or any material derived from it for commercial and/or monetary purposes should be communicated to the author and made with explicit written permission. Finally, note that the author is not liable for any direct or indirect consequential damages of any kind that arise from the use of any material in this work or from any derivatives of it in which the author is not directly involved.

\todo[inline]{Possible references to current or possible future legislation/regulations about the use of AI for healthcare applications, healthcare data, or other related topics (e.g., GDPR, HIPAA, EU AI Act, etc.)}

\section*{Ethical Considerations}  \label{ethics}
\addcontentsline{toc}{section}{Ethical Considerations}

The EU published in 2019 their `Ethic guidelines for trustworthy AI' \cite{noauthor_ethics_2019}, which are based on seven key requirements that AI systems should meet in order to be considered trustworthy. These requirements are human agency and oversight, technical robustness and safety, privacy and data governance, transparency, diversity with regard to non-discrimination and fairness, environmental and societal well-being, and accountability. 

Any AI system that is developed and deployed should meet these requirements, and this work is no exception \dots

Furthermore, the more recent EU AI Act \cite{eu_aiact_2023}\dots

% Un TFG de Ingeniería, presentado en la Escuela Politécnica Superior de la UC3M, debe incorporar un apartado de marco regulador. Conforme la matriz de evaluación del Trabajo de Fin de Grado para ingenierías, el trabajo deberá incorporar el desarrollo de uno o varios de los siguientes apartados:

% Análisis de la legislación aplicable sobre la implementación descrita en el trabajo (riesgos, responsabilidades profesionales, responsabilidades éticas, riesgos laborales, privacidad y seguridad, etc.)
% Estándares técnicos, si son aplicables (sobre tecnología desarrollada, implantada, sobre lenguajes de programación o herramientas utilizadas, etc.)
% Estudio de las cuestiones relacionadas con la propiedad intelectual e industrial de la idea (patentabilidad, protección) por ejemplo, si es un trabajo teórico.
% El contenido del apartado "Marco regulador" no debe limitarse a transcribir la normativa o fragmentos de referencias sin añadir ningún contenido propio por parte del alumno.

\chapter*{Socio-Economic Environment} \label{chap:socioeconomic}
\addcontentsline{toc}{chapter}{Socio-Economic Environment}

% Un TFG de Ingeniería, presentado en la Escuela Politécnica Superior de la UC3M, debe incorporar un apartado de entorno socio-económico. Conforme la matriz de evaluación del Trabajo de Fin de Grado para ingenierías, el trabajo deberá incorporar el desarrollo de los siguientes apartados:

% Presupuesto de la elaboración del TFG.

\phantomsection
\section*{Budget} \label{budget}
\addcontentsline{toc}{section}{Budget}
\vspace*{-0.05cm}

The estimated costs of the realization of this project include those related to the human labor and material costs associated with it.

In terms of human resources, the student author of the work, and the advisor and co-advisor of the thesis are considered as the main contributors to the project. The student author is considered to have worked part-time on the project for the duration of the work, while the advisor and co-advisor contributed a few hours per week. The salary per hour of each contributor is estimated to be €18, €30, and €60, respectively, to reflect the estimated salaries of a junior researcher, senior researcher, and university professor.

By analyzing the time planning of the project throughout its duration \todo{include Gannt chart?}, the project is estimated to have lasted approximately eight months. During this time, the author contributed an estimated total of 530 hours, while the advisor and co-advisor contributed 50 and 35 hours, respectively. 

Table \ref{tab:budget-human_resources} shows a summary of the estimated costs of human resources. Thus, we calculate a \textbf{total budget of €13140.00} associated with the labor costs of this work.

\begin{table}[H]
    \centering
    % \captionsetup[]{}
    \caption{Estimated Costs of Human Resources}
    \label{tab:budget-human_resources}
    % \hspace*{-0.8cm}
    \begin{tabular}{lrrr}
    \toprule
    {} &  \# of Hours &  Salary per hour (€) &  Total Salary (€) \\
    \midrule
    Author  &                   530 &                   18 &         9540 \\
    Co-Advisor &                35 &                    60 &         2100 \\
    Advisor &                   50 &                    30 &         1500 \\
    \bottomrule
    {} &  {} & \textbf{Total Cost:} & \textbf{€13140.00} \\
    \end{tabular}
\end{table}

\vspace{-0.4cm}

In terms of non-human resource costs, the only relevant ones are those associated with the material costs utilized throughout the work. This is because only open-source software tools (Python, LaTeX, Google Colab, etc.) were used. These tools are freely available or under permissive licenses, thus, no costs are associated with them.

In terms of material costs, we consider the following:

\begin{itemize}
    \item The personal computer from which the thesis was written and experiments designed and executed. The device consisted of a 14" Apple Macbook Pro (2021) valued at €2200 at the time of writing \cite{apple_comprar_2022}. With a 3-year depreciation of €733.33 divided by the duration of the project, this would add €489 to the material costs associated with this work.
    \item Other equipment, lab materials (e.g., paper, pens, etc.), and other miscellaneous costs are considered negligible and are not included in the budget.
\end{itemize}

% Table \ref{tab:budget-material_costs} shows a summary of the material costs associated with this project:

\begin{table}[H]
    \centering
    \captionsetup[]{}
    \caption{Estimated Material Costs}
    \label{tab:budget-material_costs}
    % \hspace*{-0.8cm}
    \begin{tabular}{lrrrr}
    \toprule
    {} &  Software \& Licenses &  Personal Computer & Miscellaneous  & \textbf{Total} \\
    \midrule
    Cost (€)  &  0 & 489 &  0 & \textbf{€489} \\
    \end{tabular}
\end{table}
% \vspace*{-0.4cm}

Thus, the total material costs associated with this work are estimated to be €489.00. If we add that to the cost of human resources, we can calculate a \textbf{total budget of €13,629.00} associated with its realization.


\section*{Socio-Economic Impact} \label{impact}
\addcontentsline{toc}{section}{Socio-Economic Impact}

% Impacto socio-económico (impacto económico, social, medioambiental, ético, etc.) esperado de la aplicación del resultado del proyecto, plan de explotación del mismo, o consideraciones sobre aspectos económicos de la temática del trabajo. 
% Para trabajos teóricos se debe detallar en qué aplicaciones prácticas podría utilizarse y qué impacto socio-económico podría generar en el sector de aplicación.


% \printbibliography
\end{document}