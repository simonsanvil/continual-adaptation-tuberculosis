\documentclass[../main.tex]{subfiles}
\graphicspath{{\subfix{../imagenes/}}}

\begin{document}
\chapter*{Regulatory Framework} \label{chap:regulatory_framework}
\addcontentsline{toc}{chapter}{Regulatory Framework}

This thesis is published under a Creative Commons license and its sole copyright belongs to the author. The work here can be shared and distributed for non-commercial purposes as long as attribution is properly credited. 
The software written for this work is also openly available
\footnote{All code developed in this work can be found under the following url: \href{}{https://github.com/simonsanvil/...}, if you have any questions, concerns, or comments, please contact the author at \href{mailto:simonsviloria@gmail.com}{simonsviloria@gmail.com}
} \change{add url to github repo}
and licensed under an \href{https://en.wikipedia.org/wiki/Apache_License}{Apache License 2.0}, which grants permission to the use, distribution, and modification of the code for commercial and non-commercial purposes (restrictions apply, see license for details). \\

\todo[inline]{Possible references to current or possible future legislation/regulations about the use of AI for healthcare applications, healthcare data, or other related topics (e.g., GDPR, HIPAA, EU AI Act, etc.)}

Finally, note that the author of this work is not liable for any direct or indirect consequential damages of any kind that arises from the use of the material in this work or from any derivatives of it.

\section*{Ethical Considerations}  \label{ethics}
\addcontentsline{toc}{section}{Ethical Considerations}

The EU published in 2019 their `Ethic guidelines for trustworthy AI' \cite{noauthor_ethics_2019}, which are based on seven key requirements that AI systems should meet in order to be considered trustworthy. These requirements are human agency and oversight, technical robustness and safety, privacy and data governance, transparency, diversity with regard to non-discrimination and fairness, environmental and societal well-being, and accountability. 

Any AI system that is developed and deployed should meet these requirements, and this work is no exception \dots

Furthermore, the more recent EU AI Act \cite{eu_aiact_2023}\dots

% Un TFG de Ingeniería, presentado en la Escuela Politécnica Superior de la UC3M, debe incorporar un apartado de marco regulador. Conforme la matriz de evaluación del Trabajo de Fin de Grado para ingenierías, el trabajo deberá incorporar el desarrollo de uno o varios de los siguientes apartados:

% Análisis de la legislación aplicable sobre la implementación descrita en el trabajo (riesgos, responsabilidades profesionales, responsabilidades éticas, riesgos laborales, privacidad y seguridad, etc.)
% Estándares técnicos, si son aplicables (sobre tecnología desarrollada, implantada, sobre lenguajes de programación o herramientas utilizadas, etc.)
% Estudio de las cuestiones relacionadas con la propiedad intelectual e industrial de la idea (patentabilidad, protección) por ejemplo, si es un trabajo teórico.
% El contenido del apartado "Marco regulador" no debe limitarse a transcribir la normativa o fragmentos de referencias sin añadir ningún contenido propio por parte del alumno.

\chapter*{Socio-Economic Environment} \label{chap:socioeconomic}
\addcontentsline{toc}{chapter}{Socio-Economic Environment}

% Un TFG de Ingeniería, presentado en la Escuela Politécnica Superior de la UC3M, debe incorporar un apartado de entorno socio-económico. Conforme la matriz de evaluación del Trabajo de Fin de Grado para ingenierías, el trabajo deberá incorporar el desarrollo de los siguientes apartados:

% Presupuesto de la elaboración del TFG.

\phantomsection
\section*{Budget} \label{budget}
\addcontentsline{toc}{section}{Budget}
\vspace*{-0.05cm}

The estimated costs of the realization of this project include those related to the human labor and material costs associated with it.

In terms of human resources, both the student author of the work and the advisor of the thesis are considered to have put hours of labor into the making of this project. We have assumed the salary of the student to be equivalent to the one of a junior engineer of about €15.00 per hour of labor, and the one of the advisor to be equivalent to that of a senior engineer of around €35.00 per hour. 

An estimate of xxx hours...

\begin{table}[H]
    \centering
    % \captionsetup[]{}
    \caption{Estimated Costs of Human Resources}
    \label{tab:budget-human_resources}
    % \hspace*{-0.8cm}
    \begin{tabular}{lrrr}
    \toprule
    {} &  \# of Hours &  Salary per hour (€) &  Total Salary (€) \\
    \midrule
    Author  &                   xxxx &                    yy &        zzzz \\
    Advisor &                     xxx &                    yy &         zzzz \\
    \bottomrule
    {} &  {} & \textbf{Total Cost:} & \textbf{€...}
    \end{tabular}
\end{table}
\vspace*{-0.4cm}

In terms of non-human resource costs, the only relevant ones are those associated with the material costs of the hardware utilized throughout the work. This is because only open-source software tools (Python, R, LaTeX) were utilized.

...

% First we take into account the university's server from which the code was written and experiments executed. According to staff, this server is valued at €12,000.00. Therefore, with a 5-year depreciation of €2,400, and considering that the duration of this work was approximately eight months, the costs associated with this asset would be €1600.

% \looseness=-1 Additionally, we include the cost of the personal computer from which the thesis was written and the remote server accessed to. This computer consisted of a 14" Apple Macbook Pro (2021) valued at €2200 at the time of writing \cite{apple_comprar_2022}. With a 3-year depreciation of €733.33 and dividing by the duration of the project, this would add an additional €489 to the material costs associated with this work.

% \begin{table}[H]
%     \centering
%     \captionsetup[]{}
%     \caption{Estimated Material Costs}
%     \label{tab:budget-material_costs}
%     % \hspace*{-0.8cm}
%     \begin{tabular}{lrrrr}
%     \toprule
%     {} &  Software \& Liceses &  University Server &  Personal Computer & \textbf{Total} \\
%     \midrule
%     Cost (€)  &  0 & 1600 &  489 & \textbf{€2089.00} \\
%     \end{tabular}
% \end{table}
% \vspace*{-0.4cm}

% Table \ref{tab:budget-material_costs} shows a summary of the material costs associated with this project. If we add both the costs of human resources and the material costs we can calculate a \textbf{total budget of €7072.00} associated with the realization of this work.


\section*{Socio-Economic Impact} \label{impact}
\addcontentsline{toc}{section}{Socio-Economic Impact}

% Impacto socio-económico (impacto económico, social, medioambiental, ético, etc.) esperado de la aplicación del resultado del proyecto, plan de explotación del mismo, o consideraciones sobre aspectos económicos de la temática del trabajo. 
% Para trabajos teóricos se debe detallar en qué aplicaciones prácticas podría utilizarse y qué impacto socio-económico podría generar en el sector de aplicación.


% \printbibliography
\end{document}