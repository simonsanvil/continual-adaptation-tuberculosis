\documentclass[../main.tex]{subfiles}


\begin{document}

%----------
%	RESUMEN Y PALABRAS CLAVE
%----------	
\renewcommand\abstractname{\large\uppercase{Abstract}}
\begin{abstract}
\thispagestyle{plain}
\setcounter{page}{3}
	
% 	% ESCRIBIR EL RESUMEN AQUÍ

% The advent of machine learning has brought significant advancements to various domains, including healthcare. In recent years, the use of machine learning (ML) in health applications has increased exponentially, resulting in improved diagnostics, personalized treatment plans, and more efficient resource allocation. However, due to the big risk of deploying ML models for high-stake decisions, together with the dynamic and data-rich nature of healthcare data, it becomes of special importance to building more robust and reliable systems for these applications. Integrating continuous, active, and online learning (CAOL) techniques in the design of machine-learning platforms can further enhance the robustness of actively deployed models against mispredictions, data/concept drifts, and misalignment. This work reviews the theoretical aspects and potential benefits of employing CAOL in developing machine-learning systems and applying it to a platform that hosts models mainly trained in the context of Tuberculosis detection. We address the key characteristics of these methods and propose some promising techniques for their implementation.


\textbf{Keywords:}
tuberculosis, machine learning, data analysis, era4tb, healthcare
% 	% Escribir las palabras clave aquí
	
	\vfill
\end{abstract}
\newpage % página en blanco o de cortesía
\thispagestyle{empty}
\mbox{}


%----------
%	DEDICATORIA
%----------	
\chapter*{Dedication}

\setcounter{page}{5}
	
% ESCRIBIR LA DEDICATORIA AQUÍ	

% To my family members, teachers, colleagues, etc...
Special thanks to my advisor, for guiding me and supporting me at every step in the making of this thesis...




	
\vfill
\newpage % página en blanco o de cortesía
\thispagestyle{empty}
\mbox{}

	
\end{document}